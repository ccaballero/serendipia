\documentclass[letter,12pt]{article}

\usepackage[T1]{fontenc}
\usepackage{lmodern}
\usepackage{textcomp}
\renewcommand*\familydefault{\sfdefault}

\usepackage[spanish]{babel}
\usepackage[utf8x]{inputenc}

\usepackage[pdftex]{graphicx}
\usepackage{pifont}
\usepackage[
pdfauthor={Carlos Caballero Burgoa},%
pdftitle={Proyecto Serendipia},%
colorlinks,%
citecolor=black,%
filecolor=black,%
linkcolor=black,%
%urlcolor=black
pdftex]{hyperref}

\usepackage{fancyhdr}
\usepackage{lastpage}
\pagestyle{fancy}

% Para la primera página
\fancypagestyle{plain}{
\fancyhead[l]{}
\fancyhead[r]{}
\fancyhead[c]{}
\renewcommand{\headrulewidth}{0.5pt}
\fancyfoot[l]{SCESI \\ Sociedad Científica de Estudiantes de Sistemas e Informática}
\fancyfoot[c]{}
\fancyfoot[r]{\thepage/\pageref{LastPage}}
\renewcommand{\footrulewidth}{0.5pt}}

% Para el resto de páginas
\lhead{Proyecto Serendipia}
\chead{}
\rhead{\includegraphics[width=0.1\textwidth]{scesi.png}}
\renewcommand{\headrulewidth}{0.4pt}
\lfoot{SCESI \\ Sociedad Científica de Estudiantes de Sistemas e Informática \\ 
\url {http://scesi.fcyt.umss.edu.bo}}
\cfoot{}
\rfoot{\thepage/\pageref{LastPage}}
\renewcommand{\footrulewidth}{0.4pt}

\title{\bf Serendipia}
\author{Carlos Caballero Burgoa} 

\begin{document}
\maketitle
\begin{center}\includegraphics[width=0.48\textwidth]{serendipia.png}\end{center}
\begin{center}\url {http://scesi.fcyt.umss.edu.bo}\end{center}
\pagebreak

\tableofcontents
\pagebreak

\section{Introducción}
Este documento se plantea como la propuesta para la puesta en marcha de un sitio web, cuya meta
es la difusión, intercambio y colaboración en proyectos de desarrollo de software en las carreras
de informática y sistemas.

Se mencionan los beneficios que conlleva escribir software que para el programador es interesante,
útil, y beneficioso; partiendo de este punto lograr ademas un beneficio para su comunidad.

Por ultimo de definen los pasos a seguir para este cometido.

\section{Antecedentes}
Es bien sabido que las herramientas que han fomentado el surgimiento de la era digital, nacieron
en gran parte como proyectos de investigación en universidades alrededor del mundo, tales
iniciativas reflejan como las ideas, esfuerzos y participación, han conseguido forjar nuestra
realidad tal y como la conocemos hoy en día.

Si bien en la Universidad Mayor de San Simón, posee una gran variedad de proyectos de investigación
en diferentes áreas de la ciencia; los aportes realizados hacia el desarrollo de software no
parecen tener grandes repercusiones sobre la sociedad en general o sobre algún sector de la
comunidad estudiantil en particular.

\section{Definición del Problema}
Se ha visto que los estudiantes tanto de la carrera de informática, como de la carrera de
ingeniería de sistemas, poseen grandes cantidades de software, producto de su propio desarrollo, ya
sea para las materias que ha cursado, o también las creados solo por pasatiempo personal; todos
esos productos subyacentes de la actividad académica, en su gran mayoría se ven desaprovechadas una
vez que el estudiante termina sus estudios superiores.

Hasta ahora el procedimiento común para la difusión de este software, se basa en el intercambio
directo del creador a los interesados del software que han oído acerca de este, dificultando mucho
la valoración sobre las programas existentes.

Todo esto tiende a la creación repetida de muchos programas simples, creados por muchas personas;
pero que nunca llegan a integrarse en herramientas mas complejas; perdiendo también la experiencia
necesaria de trabajo en equipo, carencia notoriamente apreciada entre los estudiantes.

Por lo mencionado anteriormente se define el problema como:

Las dificultades para el intercambio y la colaboración en el desarrollo de software conlleva a la
escasa creación de herramientas propias y útiles para la comunidad.

\section{Objetivo General}
Crear un canal de comunicación accesible para el intercambio de software tal que esta facilite la
creación de herramientas de mayor complejidad y funcionalidad.

\section{Objetivos Específicos}
\begin{itemize}
\item Facilitar el acceso a los programas creados dentro las carreras de informática e ingeniería
de sistemas.
\item Incrementar el uso de tales herramientas basado en la centralización de medio en el que se
pone a disposición.
\item Fomentar la creación de proyectos de desarrollo de software a gran escala.
\end{itemize}

\section{Herramientas}
Para tal cometido se considera la puesta en marcha de un sistema de control de versiones sobre un
servidor, y la definición de su conjunto de políticas de uso.

A continuación se pone de manifiesto el conjunto de pasos a realizar:

\begin{enumerate}
\item Montaje del servidor para control de versiones git.
\item Instalación y configuración de la aplicación gitolite, para el manejo de cuentas, grupos, y
repositorios.
\item Montaje de la aplicación web 'viewgit' para el acceso al software vía web.
\item Definición del conjunto de políticas de uso del sitio, de forma tal que fomente el
intercambio de software entre toda la comunidad.
\end{enumerate}

Referente al conjunto de políticas de uso, se considera imperativo restringir el uso del sitio a
herramientas que posean algún tipo de licencia permisiva de software\cite{Senabre}.

A continuación mencionamos los tipos de licencias de software que serian validas.

\begin{itemize}
\item Affero GPL.
\item Modified BSD license.
\item GNU General Public License.
\item GNU Lesser General Public License.
\end{itemize}

\section{Justificación}
Este proyecto intenta ahorrar el tiempo que las personas utilizan en de\-sa\-rro\-llar programas
propios, cuando pueden existir otras herramientas equivalentes ya creadas.

El otro objetivo a largo plazo es la creación de proyectos de software de mayor repercusión y que
gocen de la contribución de una gama amplia de personase en el ámbito local.

\begin{thebibliography}{99}
\bibitem{Senabre} Senabre Hidalgo, Enric, 2005.\\
"La colaboración en el desarrollo del software libre".\\
Disponible en el ARCHIVO del Observatorio para la CiberSociedad en
http://www.cibersociedad.net/archivo/articulo.php?art=202
\end{thebibliography}

\end{document}          
